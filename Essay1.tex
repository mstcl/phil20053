\documentclass[11pt]{article}

% Packages
\usepackage[T1]{fontenc}
\usepackage[utf8]{inputenc}
\usepackage{newpxtext}
\usepackage{newpxmath}
\usepackage[margin=1.3in]{geometry}
\usepackage{csquotes}
\usepackage{enumitem}
\usepackage[british]{babel}
\usepackage{ragged2e}
\usepackage[indentfirst=false]{quoting}
\usepackage[backend=biber,style=authoryear-ibid,pagetracker=true,autocite=footnote,ibidpage=true,loccittracker=context,firstinits=true]{biblatex}

% Bibliography
\bibliography{main_references}

% Commands
\newcommand{\customcite}[1]{\citeauthor{#1}, \citeyear{#1}}
\newcommand{\thesislabel}[2]{#2\def\@currentlabel{#2}\label{#1}}
\newenvironment{midtext}[1][2em]
{\begin{quoting}[leftmargin=#1,rightmargin=#1]\RaggedRight}
		{\end{quoting}}

% Title
\title{What are Aristotelian natural motions? How did post-Copernicans come to
	reject this concept? What is gained and what is lost in this rejection of natural
	motions?}
\author{2001747}
\date{}

% Content
\begin{document}
\maketitle

\section{Thesis}%
\label{sec:thesis}

I will argue that from the emancipation of Aristotelian natural
motions, there is a loss of metaphysical hierarchy and bias in
the conception of space (and thus cosmology), and at the same
time took part in the eventual overthrowing of the Peripatetic
methodology and doctrine in philosophical thinking and reasoning
at the time. This leads to a freer mode of thinking for the
concept of space, which helped (re)-introduce mathematical and
mechanistic elements as part of a revival of the Platonic and
Atomist methodology.

\begingroup
\begin{midtext}
	\begin{enumerate}[label=(\textbf{L})\ ]
		\item\label{Argument}
		\begin{enumerate}[label=(P\arabic*)]
			\item There is a loss of long-standing metaphysical bias and hierarchy
			      in the conception of space.
			\item This meant a gain in freedom in establishing metaphysical
			      foundations in theories of space.
			\item A new methodology could develop (and did develop) out of this
			      freedom.

			      \moveright.6in\vbox{\hrule width 5in}

			\item[(C)] My thesis.\label{item:L_conclusion}
		\end{enumerate}
	\end{enumerate}
\end{midtext}
\endgroup

\section{Introduction}%
\label{sec:introduction}

\sloppy

\section{Aristotelian natural motions}%
\label{sec:aristotelian_natural_motions}

In Aristotle's \textit{Physics}, natural motions were those
which bodies exhibit naturally, without any ``constant action of
an external agent''\autocite[18]{Cushing1998}. To determine
natural motions, according to Aristotle, one considers the
composition and proportion of the four terrestrial basic
elements (earth, air, fire, and water). Each element had its own
natural place to which they move with their natural motions. The
heavier an element is, the closer it is to the centre of the
universe, while the lighter an element is, the natural place is
found closer to the edge of the universe (Aristotle's universe
was finite). Earth's natural place is at the centre, as it is
the heaviest element, so its natural motion is downwards; while
for fire, it is upwards. Therefore, all terrestrial elements
move in vertical, straight lines. The heavenly element, aether,
had its natural motion associated with uniform circular motion,
because the heavens must be eternal and perfect.

Aristotelian philosophy was endowed with ontological explanation
(in comparison to Atomists' physical character or Platonists'
mathematics)\autocite[69]{Jammer1954} and full of dual concepts,
notably potentiality-actuality, matter-form. This is central to
understand why bodies move towards their natural place. Once
reached, bodies are in full actuality and remain there--- while
moving towards their natural place, they still have potentiality
and must continue to move. In \textit{De Caelo}, the universe is
an organism, not in the literal modern sense, but as a way to
explain that because all bodies have a goal (the final cause, or
\textit{telos}, one of the four Aristotelian causes) and all
bodies tend towards their natural goal, there must be inherent
``directional tendencies'' for all matter in the universe.

\section{Post-Copernican rejection of natural motions}%
\label{sec:post_copernican_rejection_of_natural_motions}

Aristotle's metaphysics and physics posed many problems and both
pre- and post-Copernican figures have extensively dissected and
attempted to solve these problems. With this in mind, I can only
discuss one notable argument that lead to arguably the most
compelling rejection of natural motion, an argument put forth by
Galileo. That is to say, what Galileo rejected was more subtle
than the entire Aristotelian concept of natural motions. He
wanted to challenge the dogma that a body will fall with a rate
proportional to how heavy it is\autocite[81]{Cushing1998}, i.e.\
much earth it comprises (I use the term \textit{weight}
henceforth to denote heaviness). The argument is laid out as
follows:

A lighter body $m$ falls with velocity $v$, while a heavier body
$M$ falls with velocity $V$. The connected body $\left(m +
	M\right)$ falls with velocity $v^\prime$. Now, here are the
inequalities entailed by Aristotle's physics: $m < M $ (with
respect to their weights), so $v < V$. Consider the joint
body---since $m$ contains less earth, it should retard $M$, thus
$v < v^\prime < V$ (conclusion $A$). At the same time, $\left(m
	+ M\right) > M$, so it follows that $v < V < v^\prime$
(conclusion $B$). But $A$ and $B$ contradict \textit{a priori},
except in the case where $v = V = v^\prime$ (conclusion $C$).
However, Aristotle's physics maintained as fact that $v < V$,
and as a result, Galileo's argument, a \textit{reductio ad
	absurdum}, shows that one must refute Aristotle's statement
stated above via sound mathematical
deduction\autocite[82]{Cushing1998}.

Consequently, if a body's heaviness ($W$) does not affect the
degree of tendency ($s$) it has to reach its \textit{telos},
then this is a result of faulty metaphysics on Aristotle's part.
The variant between $W$ and $s$ does not exist, so a body does
not fall (faster) because it has (more) earth, i.e.\ the
addition of earth $m$ does not result in increased rate of fall.
The metaphysical implication of this refutation meant that
Aristotelian natural motions and places would have to be given
up.

While this argument is compelling on its own, Galileo,
influenced by ``impetus theory'' of Philoponus, also developed a
system to describe the kinematics of projectile motion that also
stripped away the \textit{organismic} element in which natural
motions found its roots, and furthered the necessity to move
away from this Aristotelian doctrine. I cannot go into depth,
but consulting the critique from Philoponus alone; the cause of
natural motion is found not within space, but by the agent to
the body, and thus inherent in the moving body (which ``shows a
remarkable resemblance to the gravity suggested by
Copernicus''\autocite[57]{Jammer1954}). EXPAND THIS?

The acceptance of the arguments above meant the immediate loss
of authority in Peripatetic doctrine. For not only was his
physics challenged, but his metaphysics has been brought under
scrutiny; how can the role of \textit{telos} be emphasised if
matter, being composed by smaller earth elements which all
individually express the urge to seek its natural place, no
longer varies with the rate taken to reach its goal? Many before
him took his philosophy as dogma, and it took many attempts to
overthrow this long-standing tradition which brought along with
it deep metaphysical bias and hierarchy. This break in tradition
is clearly seen by the theories which came after; the
reintroduction of Platonic emphasis on more rigorous mathematics
in Galilean, Keplerian, Cartesian, Newtonian, Leibnizian natural
philosophy and metaphysics, alongside the many Italian
Renaissance scholars who influenced these figures. While certain
traces of Aristotelian thought remain in the theories expounded
by these contemporaries, for example, the unchallenged survival
of uniform circular orbits in Copernican and Galilean cosmology
(and Cartesian vortices), the sweeping ontological principles
and distinctive elements Aristotle put forth as part of his
metaphysics and physics were no longer upheld as dogma. Again,
consider cosmological models. If the sun was considered as a
ball of fire, and any contemporary of Aristotle, say Newton, who
published their sun-centred world system without the burden to
explain why a heavenly body, let alone one considered a ball of
fire, could be placed at the centre of the universe, then such a
burden to stay faithful to Aristotle did not exist. I hope to
have shown that there is a loss of Peripatetic authority as a
partial result of natural motions (and its metaphysical
ontology) being refuted.

% It took until Kepler's time to finally remove uniform circular
% motion of astronomical orbits. From Kepler's first law, all
% bodies in orbit with the Sun follow an elliptical path, not a
% circular one. I cannot find persuasive elements in Kepler's
% arguments, for his ontology and metaphysics were not
% revolutionary in that they contain\ldots

\section{A shift in methodology}%
\label{sec:a_shift_in_methodology}

I shall argue now that by rejecting its metaphysical grounding,
the departure of natural motions meant space is now open to new
physical and metaphysical foundations, and a gain in more solid
methodology was needed for such tasks, and indeed such a gain
was observed. However, I wish to exercise my attempt carefully
as to avoid sweeping generalisations. First, this gain in new
methodology is not the same as but definitely affected by the
loss of long-standing bias and hierarchy in the conception of
space which stemmed from Aristotelian metaphysics demonstrated
in
Section~\ref{sec:post_copernican_rejection_of_natural_motions}.
The reason why these are not the same arises from the fact that
not long before the modern physics we know of today, science and
theology were intertwined (even Newton's conception of space was
his only religiously-biased one), therefore any figure
post-Copernican who proposed new physical theories in place of
natural motions will replace its metaphysical groundings with
something else. Now, whether this ``something else'' shows a
different methodological practice (and a more effective one)
usually comes unjustly with the unavoidable input of hindsight.
Rather, it is a careful analysis of historical records at the
time which can only decide whether signs of modern scientific
methodology were shown (indeed, even this requires some form of
hindsight). Let me explain using cosmological world systems as
an example.

Aristotle viewed space and matter as inextricably linked, since
natural places differ by virtue of the basic elements, so earth,
being the heaviest, found its natural place at the centre of the
universe, and the natural motions of the terrestrial elements
took place within the first sphere, the sublunary region. Beyond
this lied the celestial sphere, the planets and stars were
considered heavenly bodies, their element being aether, and
their orbits around earth uniform and circular. Mathematicians
whose constructions stayed faithfully close to this Aristotelian
two-sphere model had to employ many workarounds in order to fit
observational data. These workarounds are called \textit{ad hoc}
devices, a pejorative term given to parts of a model introduced
in order to fit observational data\autocite[22]{Cushing1998}.

Consider the Ptolemaic world system: it was observed that
planets occasional orbit backwards before returning to their
usual direction (retrograde motion), and that they appear to
vary in brightness. To explain away these troublesome phenomena,
Ptolemy introduced epicycles (smaller circles which travel in a
bigger circle) to account for retrograde motion, the eccentric
(the centre of the bigger circle) and the equant (the point of
reference wherein the epicycles appear
uniform)\autocite[51]{Cushing1998}. Copernicus, while known for
setting the stage for the heliocentric model (sun-centred) again
after previous dismissals, also employed just as many circles as
Ptolemy did\autocite[65]{Cushing1998}, and while there were
fewer \textit{ad hoc} devices at play, on the criterion of
economy and simplicity, it was not a much better construction
than Ptolemy's. Copernicus still modeled his theory to fit
observational data\autocite[67]{Cushing1998}, and any data taken
were not very accurate either, and as mentioned in
Section~\ref{sec:post_copernican_rejection_of_natural_motions},
his (and Galileo's) theory also retained Aristotelian circular
motion. At the time, on unbiased ground, it was difficult to
decide between the two rivalling models, due to observations
being limited to the naked eye, to see whether such a
construction \textit{really} explains the actual motion of
bodies. The discrimination against the Copernican model stemmed
from earth-centred bias of authority at the time which still
held onto its Aristotelian doctrines. Therefore, to completely
remain unbiased to the hindsight of scientific modernity, any
model which, no matter how close it is to the present accepted
theory, cannot show stronger methodological background if it was
arrived at with \textit{ad hoc} devices.

On the other hand, a new methodology was recorded. The invention
of the telescope lead Galileo to make observations which showed
many problems with the Ptolemaic model. As well as not showing
bias for a Copernican model\autocite[142]{Cushing1998}, he
argued with evidence and clear reasoning why such a model should
be preferred. Merits were to be given to Galileo, who argued for
the ``superiority of reason and
observation''\autocite[141]{Cushing1998} and demonstrated a
methodology different to those who came before him. Note that as
part of the series of arguments against Aristotelian straight
natural motion, experiments with objects rolling down inclined
planes were performed and the method detailed. Kepler, equipped
with Brahe's data, also arrived at his First Law, where bodies
now travel on elliptical paths, and this signified the death of
all Aristotelian natural motions. Therefore, with historical
evidence, I put forth my claim that new theories of motion and
space proposed in place of Aristotle's natural motions and
places lead to a different and better methodology.

% Post-Galileo or even post-Philoponus, even without metaphysical
% grounding for natural motions, the historical climate perhaps
% played a role in delaying a more thoughtful analysis of orbital
% motions. If straight line vertical natural motion can be easily
% demonstrated as flawed, what justified the circular national
% motion of bodies in the celestial sphere?

% The inevitable role of hindsight comes into play here; because
% modern astronomy now has one triumphant model, which is
% different from Ptolemy's and Copernicus', it seems a natural
% move \textit{prima facie} to consider epicycles and the rest as
% \textit{ad hoc} mathematical devices. Parts of theories that
% still remain in modern physics (sun as centre) take precedence
% and are considered guilt-free as a creative or clever
% anticipation, even though they were decided perhaps for
% completely different reasons, with few to none overlapping
% justification.

% However,

% Keeping in mind that it is ``overly
% simplistic''\autocite[76]{Cushing1998} to view there was no
% challenge to the dogmatic Peripatetic tradition prior to figures
% of the Renaissance (Copernicus and his contemporaries), I
% discuss the theories of these latter figures only due to their
% significance in driving the momentum of the Copernican
% revolution, not because they were the first to propose such
% ideas.

% Aristotelian space is anisotropic, i.e.\ there is a
% differentiation between up and down due to the natural motions
% of the basic elements. However, the rejection of this concept
% meant space can now be isotropic---the same in all directions.
% If space is to remain anisotropic, there must a set of more
% convincing metaphysical laws that necessitate the distinction
% and hierarchy of directions.

\section{Conclusion}%
\label{sec:conclusion}

\printbibliography % chktex 1
\end{document}
