\documentclass[12pt]{article}

% Packages
\usepackage[T1]{fontenc}
\usepackage[utf8]{inputenc}
\usepackage{newpxtext}
\usepackage{newpxmath}
\usepackage[margin=1.3in]{geometry}
\usepackage{csquotes}
\usepackage{enumitem}
\usepackage[british]{babel}
\usepackage{ragged2e}
\usepackage[indentfirst=false]{quoting}
\usepackage[backend=biber,style=authoryear-ibid,pagetracker=true,autocite=footnote,ibidpage=true,loccittracker=context,firstinits=true]{biblatex}

% Bibliography
\bibliography{main_references}

% Commands
\newcommand{\customcite}[1]{\citeauthor{#1}, \citeyear{#1}}
\newcommand{\thesislabel}[2]{#2\def\@currentlabel{#2}\label{#1}}
\newenvironment{midtext}[1][2em]
{\begin{quoting}[leftmargin=#1,rightmargin=#1]\RaggedRight}
		{\end{quoting}}

% Title
\title{What are Aristotelian natural motions? How did post-Copernicans come to
	reject this concept? What is gained and what is lost in this rejection of natural
	motions?}
\author{}
\date{}

% Content
\begin{document}
\maketitle

\section{Introduction}%
\label{sec:introduction}

\sloppy

In this essay, I will start by explaining the Aristotelian
concept of natural motions, and then provide arguments on how
one of the post-Copernican scholars, Galileo, rejected this
notion. I will then argue that resulting from the emancipation
of natural motions are (1) a loss in the role metaphysical
purpose in theories of nature, pertaining to the space, motion
and cosmology, and (2) a gain in a significant shift in
philosophical methodology away from self-evident axiomatic
reasoning to quantitative and mathematical investigation.

\section{Aristotelian natural motions}%
\label{sec:aristotelian_natural_motions}

In Aristotle's \textit{Physics}, natural motions were those
exhibited by bodies without the need of ``constant action of an
external agent''\autocite[18]{Cushing1998}. In fact, the concept
of natural motions perhaps defined what it means to be one of
the four terrestrial basic elements (earth, water, air, and
fire, in ascending order by weight)\autocite[75]{Barbour1989}.
To start, each element had its own natural place to which they
move with their natural motions. The heavier an element is, the
more it travels downwards, so it is found close to the centre of
the universe, while the lighter an element is, the more it
travels upwards, and we find it closer to the edge of the
universe (Aristotle's universe was finite). Earth's natural
place is at the centre, as it is the heaviest element; while
fire moves upwards. Therefore, all terrestrial elements move in
vertical, straight lines. The heavenly element, aether or
quintessence, had its natural motion associated with uniform
circular motion, because the heavens must be eternal and
perfect.

Aristotelian philosophy came with inextricable metaphysical
foundation (in comparison to the emphasis on physical character
by Atomists or mathematics by
Platonists)\autocite[69]{Jammer1954} and full of dual concepts,
notably potentiality-actuality, matter-form. This is central to
understand why bodies move towards their natural place. Once
reached, bodies are in full actuality and remain there--- while
moving towards their natural place, they still have potentiality
and must continue to move. As an aside, we might be able to
appreciate how this anticipated the modern concepts of potential
energy and actual, or kinetic energy. What is no longer seen in
modernity, however, is this doctrine in Aristotle's \textit{De
	Caelo}: the universe as an organism, not in the literal
animalistic sense, but as a way to explain that because all
bodies have a goal (the final cause, or \textit{telos}, one of
the Aristotle's four causes, the rest being material, formal and
efficient) and all bodies tend towards their natural goal, there
must be inherent ``directional tendencies'' for all matter in
the universe.

\section{Post-Copernican rejection of natural motions}%
\label{sec:post_copernican_rejection_of_natural_motions}

Aristotle's metaphysics and physics posed many problems and both
pre- and post-Copernican figures have extensively dissected and
attempted to solve these problems. With this in mind, I can only
discuss one notable argument that led to arguably the most
compelling rejection of natural motion, an argument put forth by
Galileo. That is to say, what Galileo rejected was more subtle
than the entire Aristotelian concept of natural motions. He
wanted to challenge the dogma that a body will fall with a rate
proportional to how heavy it is\autocite[81]{Cushing1998}, i.e.\
much earth it comprises (I use the term \textit{weight}
henceforth to denote heaviness). The argument is laid out as
follows:

A lighter body $m$ falls with velocity $v$, while a heavier body
$M$ falls with velocity $V$. The connected body $\left(m +
	M\right)$ falls with velocity $v^\prime$. Now, here are the
inequalities entailed by Aristotle's physics: $m < M $ (with
respect to their weights), so $v < V$. Consider the joint
body---since $m$ contains less earth, it should retard $M$, thus
$v < v^\prime < V$ (conclusion $A$). At the same time, $\left(m
	+ M\right) > M$, so it follows that $v < V < v^\prime$
(conclusion $B$). But $A$ and $B$ contradict \textit{a priori},
except in the case where $v = V = v^\prime$ (conclusion $C$).
However, Aristotle's physics maintained as fact that $v < V$,
and as a result, Galileo's argument, a \textit{reductio ad
	absurdum}, shows that one must refute Aristotle's statement
stated above via sound mathematical
deduction\autocite[82]{Cushing1998}.

Consequently, if a body's heaviness ($W$) does not affect the
degree of tendency ($s$) it has to reach its \textit{telos},
then this is a result of faulty metaphysics on Aristotle's part.
The variant between $W$ and $s$ does not exist, so a body does
not fall (faster) because it has (more) earth, i.e.\ the
addition of earth $m$ does not result in increased rate of fall.
The metaphysical implication of this refutation meant that
Aristotelian natural motions and places would have to be given
up.

While this argument is compelling on its own, natural motions
also struggled to explain forced motion (those acted on
externally), and Aristotle used what is known pejoratively as an
\textit{ad hoc} device to explain away why an arrow might keep
moving horizontally after it has left the archer's bow. In
essence, the organic nature of space meant the objects
comprising the medium in which a body moves through with forced
motion somehow `closes in' at the back of the projectile to keep
it along its path. In response, Philoponus posed: why can't we
displace object by pushing the air around and behind it? The
mitigation was to propose instead that the cause of natural
motion is found not within space, but by the agent to imparting
some `force' to the body, and thus remains in the moving body
(which ``shows a remarkable resemblance to the gravity suggested
by
Copernicus''\autocite[57]{Jammer1954}\autocite[156]{Cushing1998}).
Galileo, influenced by this ``impetus theory'' of Philoponus,
went on to develop a system which describe the kinematics of
projectile motion in a completely mechanistic manner. It also
explains well why the path of projectiles is parabolic and not
circular, as previously assumed. This construction completely
stripped away the \textit{organismic} element in which natural
motions found its roots, and furthered the necessity to reject
and move away from this Aristotelian doctrine.

% The acceptance of the arguments above meant the immediate loss
% of authority in Peripatetic doctrine. For not only was his
% physics challenged, but his metaphysics has been brought under
% scrutiny; how can the role of \textit{telos} be emphasised if
% matter, being composed by smaller earth elements which all
% individually express the urge to seek its natural place, no
% longer varies with the rate taken to reach its goal? Many before
% him took his philosophy as dogma, and it took many attempts to
% overthrow this long-standing tradition which brought along with
% it deep metaphysical bias and hierarchy. This break in tradition
% is clearly seen by the theories which came after; the
% reintroduction of Platonic emphasis on more rigorous mathematics
% in Galilean, Keplerian, Cartesian, Newtonian, Leibnizian natural
% philosophy and metaphysics, alongside the many Italian
% Renaissance scholars who influenced these figures. While certain
% traces of Aristotelian thought remain in the theories expounded
% by these contemporaries, for example, the unchallenged survival
% of uniform circular orbits in Copernican\autocite[148]{Kuhn1957}
% and Galilean cosmology (and Cartesian vortices), the sweeping
% ontological principles and distinctive elements Aristotle put
% forth as part of his metaphysics and physics were no longer
% upheld as dogma. Again, consider cosmological models. If the sun
% was considered as a ball of fire, and any contemporary of
% Aristotle, say Newton, who published their sun-centred world
% system without the burden to explain why a heavenly body, let
% alone one considered a ball of fire, could be placed at the
% centre of the universe, then such a burden to stay faithful to
% Aristotle did not exist. I hope to have shown that there is a
% loss of Peripatetic authority as a partial result of natural
% motions (and its metaphysical ontology) being refuted.

\section{Losses}%
\label{sec:losses}

By accepting this rejection, the entire body of Aristotelian
metaphysical doctrine could be brought under scrutiny. Those who
held onto the shaky foundations of his theories did so not out
of philosophical justification but was a matter of theological
power struggle. While not all Aristotelian concepts were flawed,
and some, if further developed, might have achieved what was
communicated much later in the scientific
revolution\autocite[77]{Barbour1989}, there were two conflicting
mechanism at play: either towards teleology (explanation as a
function of purposes or ends) and theology, or towards careful
quantitative experimentation\autocite[77]{Barbour1989}.
Theological climate at the time meant the former was
``intimately entangled'' with astronomy. The upshot of this
meant the former direction was preferred, and led to the
ridiculing of `heretical' models of
nature\autocite[86]{Kuhn1957}. Nevertheless, with this
rejection, what was challenged was not only a long-standing
theological dogma, but also the metaphysical emphasis on
\textit{telos}. I shall argue for this loss which resulted from
the rejection of natural motions.

First, how can the role of \textit{telos} be emphasised if
matter, being composed by smaller earth elements which all
individually express the urge to seek its natural place, no
longer varies with the rate taken to reach its goal? If
mechanistic and mathematical relations can describe the motion
of a projectile, in the case of Galileo, or the world-system, in
the case of heliocentric overturn of natural places, better than
Aristotle's theories, then where is the causal role of
\textit{telos} in nature? Although theistic and mystic elements
have yet to leave natural philosophy (even Newton's conception
of space was religiously motivated, as he was influenced by
religious elements in More's
philosophy\autocite[157]{Cushing1998}\autocite[113]{Jammer1954})
until the logical positivist revolution took over the domains of
natural philosophy (the view that statements can only have
meaning if they can be demonstrated through direct observation
or logical proof), \textit{telos} as a final cause did not find
itself revived in the post-Copernican theories and signified the
eventual death of a spatial ``hierarchy of
values''\autocite[82-3]{Jammer1954}. Take Galilean
invariance---physical laws are unchanged in two systems with
relative motion between each other, this would seem to imply
that the degree of \textit{telos} is changed under a simple
Galilean transformation between frames of reference. Consider a
barrel next to me, both on a moving ship, would appear
stationary in my frame (it has reached its natural place), but
indeed is moving at the ship's speed with respect to the waves
(which might have some vertical motion), and the waves at some
relative speed with respect to the earth (which moves in a
complex manner, a combination of both vertical and horizontal),
so on and so forth. So with each subsequent transformation, the
rate at which the barrel travels to the centre of the universe
is different, implying either that its matter has to be changing
(which is not), or that its \textit{telos} is. As the final
cause, however, \textit{telos} ultimately emenates from the
Prime Mover, which is the teleological divine entity responsible
for world order. The Prime Mover is absolutely actual and never
potential, thus cannot change.

What's more, even though Leibniz came the closest to the
Aristotelian concept of hylomorphism (duality of matter and
form) by imbuing material bodies with \textit{vis viva}---the
quantity $mv^2$ known otherwise as living force---even such a
quantity is to be determined from another mechanistic quantity:
velocity squared (mass $m$ here is left open for interpretation;
Leibniz's monadic philosophy will not be discussed here, but it
was likened to Atomism\autocite[64]{Jammer1954} rather than
Aristotelianism or Scholasticism). The causal nature of
\textit{telos} was nowhere to be found. For many of these
theories, there was a shift in nature, from being organismic to
being mechanistic\autocite[19]{Cushing1998}, and the four causes
collapsed to just one efficient cause. While it is possible to
provide more accounts of new, non-Aristotelian metaphysical
foundations, I hope this is ample evidence to show that the
concept of \textit{telos} was buried alongside natural motions.

\section{Gains}%
\label{sec:gains}

Complementary to this loss was the slow shift towards the other
direction in the development of natural philosophy. Whereas the
medieval philosophers prioritised metaphysical foundation to
appease to religious orthodoxy, the rejection of natural motions
was arrived at with increasing preference towards ``quantitative
investigation''\autocite[77]{Barbour1989}, notwithstanding
rigorous mathematical and logical argumentation. While
mathematical constructions appeared under the Peripatetic
authority, they were ridden with problems that would only be
covered up by the use \textit{ad hoc} devices. An example is
seen with the complicated Aristotelian-Ptolemaic geocentric
world-system of off-centred orbits and circles-within-circles.
Even when it was not common for some Aristotelian thinking to
remain until some creative input broke free of the old
tradition, as seen with uniform circular motion as the natural
motion of planets in Copernican, Galilean and Cartesian physics
(Kepler's First Law about elliptical orbits of bodies around the
sun which finally put an end to the circle), many figures
drifted away from axiomatic investigation (emphasis on
``self-evident'' axioms and ``cursory''
observation\autocite[34]{Cushing1998}) and onto inductive
(generalisations from more careful observations) and
retroductive (going from theories to hypotheses or predictions
that are validated or falsified with further observations)
investigation\autocite[35]{Cushing1998}. Let me provide further
evidence.

%\section{A shift in methodology}%
%\label{sec:a_shift_in_methodology}

%However, I wish to exercise my attempt carefully as to avoid
%sweeping generalisations. First, this gain in new methodology is
%not the same as but definitely affected by the loss of
%long-standing metaphysical bias demonstrated in
%Section~\ref{sec:post_copernican_rejection_of_natural_motions}.
%The reason why these are not the same arises from the fact that
%not long before the modern physics we know of today, science and
%theology were intertwined (even Newton's conception of space was
%his only religiously-biased one), therefore any figure
%post-Copernican who proposed new physical theories in place of
%natural motions will replace its metaphysical groundings with
%something else. Now, whether this ``something else'' shows a
%different methodological practice (and a more effective one)
%usually comes unjustly with the unavoidable input of hindsight.
%Rather, it is a careful analysis of historical records at the
%time which can only decide whether signs of modern scientific
%methodology were shown (indeed, even this requires some form of
%hindsight). Let me explain using cosmological world systems as
%an example.

%Aristotle viewed space and matter as inextricably linked, since
%natural places differ by virtue of the basic elements, so earth,
%being the heaviest, found its natural place at the centre of the
%universe, and the natural motions of the terrestrial elements
%took place within the first sphere, the sublunary region. Beyond
%this lied the celestial sphere, the planets and stars were
%considered heavenly bodies, their element being aether, and
%their orbits around earth uniform and circular. Mathematicians
%whose constructions stayed faithfully close to this Aristotelian
%two-sphere model had to employ many workarounds in order to fit
%observational data. These workarounds are called \textit{ad hoc}
%devices, a pejorative term given to parts of a model introduced
%in order to fit observational data\autocite[22]{Cushing1998}.

% Consider the Ptolemaic world system: it was observed that
% planets occasional orbit backwards before returning to their
% usual direction (retrograde motion), and that they appear to
% vary in brightness. To explain away these troublesome phenomena,
% Ptolemy introduced epicycles (smaller circles which travel in a
% bigger circle) to account for retrograde motion, the eccentric
% (the centre of the bigger circle) and the equant (the point of
% reference wherein the epicycles appear
% uniform)\autocite[51]{Cushing1998}. Copernicus, while known for
% setting the stage for the heliocentric model (sun-centred) again
% after previous dismissals, also employed just as many circles as
% Ptolemy did\autocite[65]{Cushing1998}, and while there were
% fewer \textit{ad hoc} devices at play, on the criterion of
% economy and simplicity, it was not a much better construction
% than Ptolemy's. Copernicus still modeled his theory to fit
% observational data\autocite[67]{Cushing1998}, and any data taken
% were not very accurate either, and as mentioned in
% Section~\ref{sec:post_copernican_rejection_of_natural_motions},
% his (and Galileo's) theory also retained Aristotelian circular
% motion. At the time, on unbiased ground, it was difficult to
% decide between the two rivalling models, due to observations
% being limited to the naked eye, to see whether such a
% construction \textit{really} explains the actual motion of
% bodies. The discrimination against the Copernican model stemmed
% from earth-centred bias of authority at the time which still
% held onto its Aristotelian doctrines. Therefore, to completely
% remain unbiased to the hindsight of scientific modernity, any
% model which, no matter how close it is to the present accepted
% theory, cannot show stronger methodological background if it was
% arrived at with \textit{ad hoc} devices.

The invention of the telescope meant Galileo made observations
which showed many problems with the Ptolemaic geocentric model,
as well as lunar craters and sunspots, which were direct
conflicts to Aristotelian order and symmetry. At the same time,
this meant ``countless evidence'' for the Copernican
heliocentric model\autocite[219]{Kuhn1957}. While Galileo did
not show bias for a Copernicanism\autocite[142]{Cushing1998}, he
argued with evidence and clear reasoning why such a model should
be preferred. Merits were to be given to Galileo, who supported
the ``superiority of reason and
observation''\autocite[141]{Cushing1998}. If a theory does not
correspond with observation, it does not suggest the use of
arbitrary \textit{ad hoc} solutions, but perhaps the abandonment
of the theory itself, especially if it carries metaphysical
bias. What's more, as part of the series of arguments against
Aristotelian straight natural motion, experiments with objects
rolling down inclined planes were performed and a mathematical
relation was arrived at---$v \propto t^2$ (as opposed to $v
	\propto m$), as well as those for projectile motions mentioned
above. While innovative theories such as Galileo's (and
Kepler's, not mentioned here) were not purported without
opposition, their eventual assimilation meant gain in a new
priority in philosophical investigation; not in ontological
origins, but in careful quantitative observation and
experimentation.

%Kepler,
%with astronomical data from his predecessor Brahe, which at the time was extremely accurate, also arrived at his three mathematical laws for planetary orbits. These laws themselves were picked up by Newton, whose own laws of motions were used to derive back the Keplerian laws, 

% Post-Galileo or even post-Philoponus, even without metaphysical
% grounding for natural motions, the historical climate perhaps
% played a role in delaying a more thoughtful analysis of orbital
% motions. If straight line vertical natural motion can be easily
% demonstrated as flawed, what justified the circular national
% motion of bodies in the celestial sphere?

% The inevitable role of hindsight comes into play here; because
% modern astronomy now has one triumphant model, which is
% different from Ptolemy's and Copernicus', it seems a natural
% move \textit{prima facie} to consider epicycles and the rest as
% \textit{ad hoc} mathematical devices. Parts of theories that
% still remain in modern physics (sun as centre) take precedence
% and are considered guilt-free as a creative or clever
% anticipation, even though they were decided perhaps for
% completely different reasons, with few to none overlapping
% justification.

% However,

% Keeping in mind that it is ``overly
% simplistic''\autocite[76]{Cushing1998} to view there was no
% challenge to the dogmatic Peripatetic tradition prior to figures
% of the Renaissance (Copernicus and his contemporaries), I
% discuss the theories of these latter figures only due to their
% significance in driving the momentum of the Copernican
% revolution, not because they were the first to propose such
% ideas.

% Aristotelian space is anisotropic, i.e.\ there is a
% differentiation between up and down due to the natural motions
% of the basic elements. However, the rejection of this concept
% meant space can now be isotropic---the same in all directions.
% If space is to remain anisotropic, there must a set of more
% convincing metaphysical laws that necessitate the distinction
% and hierarchy of directions.

\section{Conclusion}%
\label{sec:conclusion}

I have answered the question by explaining the concept of
natural motions, which were the defining properties for matter
in the form of the basic elements, then argued how this was
rejected from two Galilean arguments. From the abandonment of
this concept, I argued that this resulted in the loss of the
Aristotelian doctrine of endowing space with metaphysical
\textit{telos} notwithstanding the overturning of Peripatetic
dogmatic way of thinking. At the same time, this also paved way
to newer methodology, which placed less emphasis on metaphysics
and theology, while involving more rigorous quantitative
investigation and experimentation, as well as regarding strong
mathematical and logical reasoning as championing over
metaphysical foundations.

\printbibliography % chktex 1
\end{document}
