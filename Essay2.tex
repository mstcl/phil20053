\documentclass[11pt]{article}

% Packages
\usepackage[T1]{fontenc}
\usepackage[utf8]{inputenc}
\usepackage{newpxtext}
\usepackage[margin=1.3in]{geometry}
\usepackage{csquotes}
\usepackage{enumitem}
\usepackage[british]{babel}
\usepackage{ragged2e}
\usepackage[indentfirst=false]{quoting}
\usepackage[backend=biber,style=authoryear-ibid,pagetracker=true,autocite=footnote,ibidpage=true,loccittracker=context,firstinits=true]{biblatex}

% Bibliography
\bibliography{main_references}

% Commands
\newcommand{\customcite}[1]{\citeauthor{#1}, \citeyear{#1}}
\newcommand{\thesislabel}[2]{#2\def\@currentlabel{#2}\label{#1}}
\newenvironment{midtext}[1][2em]
{\begin{quoting}[leftmargin=#1,rightmargin=#1]\RaggedRight}
		{\end{quoting}}

% Title
\title{Is matter necessary for the existence of space and time? Discuss with
	reference to two or more thinkers.}
\author{2001747}
\date{}

% Content
\begin{document}
\maketitle
\section{Thesis}%
\label{sec:introduction}

\sloppy

\begingroup
\begin{midtext}
	\begin{enumerate}[label=(\textbf{F})\ ]
		\item\label{RDT}\begin{enumerate}[label=(P\arabic*)]
			\item If an entity is necessary (and sufficient) for the existence of
			      space, then space is relational. \label{item:P1}
			\item If space is relational, then matter is necessary (and
			      sufficient) for the existence of space. \label{item:P2}
			\item If space is substantival, then it is absolute. \label{item:P3}
			\item Space is not absolute. \label{item:P4}
			\item[(C1)] Space is not substantival, or in other words, it is relational. \label{item:C1}

				\moveright.6in\vbox{\hrule width 3in}

			\item[(C2)] Matter is necessary for the existence of
				space.\label{item:Conclusion}
		\end{enumerate}
	\end{enumerate}
\end{midtext}
\endgroup

\section{Premise 2}%
\label{sec:premise_2}

I shall argue that if space is a relational entity, and not a
substantival one, then space necessarily depends on matter for
its existence. That is to say, a system of material bodies with
spatial relation, or ``successions of situation'', is sufficient
and necessary in defining space as a relational entity.

Before starting, I would like to assume the naive substantial
distinction between the physical and the mental. The reason
being if the latter collapses into the former, then my premise
is true by definition. As a caveat, I shall not attempt here to
demonstrate why the former cannot collapse into the latter, but
I can claim such a view ultimately defeats the point of
answering this question.

First, if space, as abstracted from everything, is a relational
entity, then the propositional properties derived from a set of
sentences describing a system of bodies in space would be
relational properties. If ``I am next to the computer'' is such
a system of bodies, then ``being next to'' is the relational
property derived from the system. Now, if I remove the subject
and object particles from the sentence, and replace them with
$X$ and $Y$ respectively, then we have the sentence ``X is next
to $Y$''. If I attempt to substitute $X$ and $Y$ with something,
then I \textit{prima facie} do so with physical objects---those
comprises of matter. I claim that such an action is ostensible
proof that only objects of matter can substitute $X$ or $Y$ in
this sentence. While such a claim appear terse, consider the
opposing case using mental images. If I now sit in park and
think of an image of myself being next to my computer, then
while I can utter the sentence ``I am next to the computer'',
such a sentence will make no sense unless I clarify that ``the
concept of myself is next to the concept of a computer''. But
even then, such a sentence is meaningless. If two concepts can
be said to be next to one another, then they can also be said to
be on top, under, in front, etc., without any means to
differentiate between them---they are mere utterances.
Therefore, while it might be painfully trivial, if spatial
relations can be ascribed to both physical or mental bodies,
only the first will have any worthwhile propositional content.
This proves my first premise.

\section{Premise 3}%
\label{sec:premise_3}

To clarify, by``substantival'', I mean a classification of space
which considers it as a substance, or in other words, that which
exists on its own. Being substantival necessarily entails that
matter is independent from space, by definition. If space is not
a substance, then it is relational in nature.
%(I will not involve super-substanvialism in my discussion as
%this involves showing matter is not a substance).

A second classification involves absolutism. If space is
absolute, and let's take Euclidean three-dimensional geometrical
infinite space as a representation, then there exists an
immovable and distinct reference point
$\left(x=0,y=0,z=0\right)$ which determines the origin of all
future translated Euclidean space. All points are therefore
relative to this origin.

Substantival space is absolute. If space is a substance, then it
cannot be determined by other entities, including its origin.
However, the reverse does not hold---space can be absolute yet
not a substance. Newton brought in the role of God, and likened
absolute space to God's sensorium, and hence space cannot be a
substance if its existence depends on at least one other entity
(a Deity).

\section{Premise 4}%
\label{sec:premise_4}

I shall now take on the task of evaluating Leibniz's use of the
Principle of Identity of the Indiscernibles (PII onwards)
against absolute space can subsequently. I will claim his
attempt as ultimately successful.

PII, in the strong form, which I will utilise, holds that no two
indiscernible entities can share all their intrinsic
properties\autocite[146]{Khamara1988}. An intrinsic property is
one in which entities can have in virtue of they way they are,
and not in relation (or lack thereof) to other
entities\autocite[144]{Khamara1988}. For PII, the ``stronger''
its formulation is, the stricter the conditions become for
entities to be considered distinct. If $A$ and $B$ share
non-identical sets of non-intrinsic properties $N_1 \neq N_2$ ,
but identical sets of intrinsic properties $I_1 = I_2$, no
matter how small the intrinsic sets or how large the
non-intrinsic sets are, if this latter condition holds, then $A$
and $B$ are indiscernible, hence it follows from PII that a
universe with both $A$ and $B$ are not logically possible.

\subsection{Two arguments against absolute space}%
\label{sub:two_arguments_against_absolute_space}

PII was established and used by Leibniz to demonstrate against
Newtonian absolute space. Newton, in his \textit{Principia},
espoused for a theory where space is ``immovable..'' CITE. This
was put forth in response to problems found in Descartes'
relative space. The biggest problem was that rest and uniform
motion were both involved circularly in Descartes' definition of
motion, so either they have to be treated as two distinct
concepts (from which Newton picked up) or the Cartesian
principle of inertia has to be rejected (Leibniz's move in order
to defend relative space). For Leibniz, treatment of space and
matter cannot be separated because space for him was compared to
a genealogical tree CITE (jammer/cushing), and the action of
abstraction to space presupposes a system of material bodies. In
setting up the scene, Leibniz's proposed two \textit{reductio ad
	absurdum} to refute absolute space, alongside %theological arguments using the Principle of Sufficient Reason (henceforth PSR---everything which exists must have sufficient reason, cause, or ground).

Consider space is absolute, and a possible world U1 with finite
or infinite material bodies, and possible world U2 with the same
finite or infinite material bodies. U2 is numerically distinct
from U1, even if all the bodies are relationally identical. Then
from PII, U2 is indiscernible from U1. But this leads to a
contradiction. Therefore, we must forgo absolute space.

The above argument comes from translating space (like adding a
constant amount to the origin), so numerical distinctness but no
relational differentation results. In two-dimensional Euclidean
space, consider a triangle with base and height both of length
1, with coordinates (0,0), (0,1) and (1,1). If I now translate
the origin of this space from (0,0) to (2,2), then the triangle
now has coordinates (2,2), (2,3) and (3,3). The numerical
relation between the vertices remain unchanged, so are the base
and height. Thus, the intrinsic property, a unit right-angled
triangle arranged like so, is invariant.

The second argument follows the exact same logical structure,
but the translation has been replaced by a reflection along the
vertical. The argument is simple, if the entire universe at
present is flipped (the universe in the mirror), then would I
observe any intra-world difference? Leibniz argued from PII for
both translation and reflection that absolute space must be
refuted.

\subsection{A third argument}%
\label{sub:a_third_argument}

With the presence of some bodies then, space is non-absolute.
What about the case in which a possible world devoid of matter
is considered instead? Here, Leibniz would argue that because
absolute space can be divided into arbitrarily small parts, each
part identical to one another, then from PII there is only one
such arbitrarily small part. So the entire universe is one such
small part. From this absurd consequence, Leibniz rejected
absolute space. Khamara argued here that the use of PII in this
scenario is intra-world (parts of a world) and not inter-world
(between worlds, like the two arguments above), and since
abusing PII intra-world leads to inconsistencies, Khamara
dismissed the validity of this argument. CITE

However, if we consider (P3) of the thesis, then notice that by
considering a universe devoid of any matter, we already
presuppose it is substantivalist, and hence by \textit{modus
	tollens} it is absolute. Hence, trying to prove relative space
in the case where absolute space is true is a logical
contradiction (since the two are mutually incompatible), so the
argument as a whole is unnecessary. The two arguments from
translation and reflection are sufficient for the refutation.
That is to say, to retain the efficacy of this argument, we
would have to refute (P3). I cannot see a way to do so. Space as
a substance cannot move into any other ``space'', thus it is
immovable and that is the definition of absolute space.

But if we do not retreat the argument, then it is possible to
argue the following. Using PII intra-world leads to absurd
consequences, such as the case where a system of $N$
intrinsically identical spheres placed in arbitrary locations is
the same as a system of $N+1$ also identical spheres (and
identical to the first system) placed in another arbitrary set
of locations. While this may suggest this flavour of PII is too
strict---we want \textit{prima facie} a principle capable of
capturing the most intrinsic information and PII does not suit
this role, it may also suggest that PII happens to
\textit{describe} the physical world, where no such ideal system
can exist where multiple bodies are considered intrinsically
identical (assumed with the hindsight of modern physics). While
this boils down to discussing the act of using physically
impossible thought experiments to test logic, I will not discuss
this further.

My point remains that absolute space is sufficiently refuted
from the two arguments from translation and reflection alone.

\section{Arriving at the conclusion}%
\label{sec:arriving_at_the_conclusion}

From here, it is a matter of logical syllogism; if space is not
absolute, or (P4) is true, then space is not substantival
(principle of \textit{modus tollens}). If space is not
substantival then it is relational, and if it is relational,
then applying \textit{modus ponens} on (P2), which I have argued
to be true, we can henceforth argue that matter is necessary for
the existence of space.

\section{What about time?}%
\label{sec:what_about_time?}

\printbibliography % chktex 1
\end{document}
