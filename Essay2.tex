\documentclass[11pt]{article}

% Packages
\usepackage[T1]{fontenc}
\usepackage[utf8]{inputenc}
\usepackage{newpxtext}
\usepackage[margin=1.3in]{geometry}
\usepackage{csquotes}
\usepackage{enumitem}
\usepackage[british]{babel}
\usepackage{ragged2e}
\usepackage[indentfirst=false]{quoting}
\usepackage[backend=biber,style=authoryear-ibid,pagetracker=true,autocite=footnote,ibidpage=true,loccittracker=context,firstinits=true]{biblatex}

% Bibliography
\bibliography{main_references}

% Commands
\newcommand{\customcite}[1]{\citeauthor{#1}, \citeyear{#1}}
\newcommand{\thesislabel}[2]{#2\def\@currentlabel{#2}\label{#1}}
\newenvironment{midtext}[1][2em]
{\begin{quoting}[leftmargin=#1,rightmargin=#1]\RaggedRight}
		{\end{quoting}}

% Title
\title{Is matter necessary for the existence of space and time? Discuss with
	reference to two or more thinkers.}
\author{2001747}
\date{}

% Content
\begin{document}
\maketitle
\section{Thesis}%
\label{sec:introduction}

\sloppy

\begingroup
\begin{midtext}
	\begin{enumerate}[label=(\textbf{F})\ ]
		\item\label{RDT}\begin{enumerate}[label=(P\arabic*)]
			\item Matter is the entity in which a relational space necessarily
			      depends on for its existence.\label{item:P2}
			\item Space is (not) a relational entity and its existence
			      metaphysically does (does not) depend on another
			      entity.\label{item:P1}

			      \moveright.6in\vbox{\hrule width 3in}

			\item[(C)] Matter is (not) necessary for the existence of
				space.\label{item:Conclusion}
		\end{enumerate}
	\end{enumerate}
\end{midtext}
\endgroup

\section{Premise 1}%
\label{sec:premise_1}

I shall argue that if space is a relational entity, and not a
substantival one, then space necessarily depends on matter for
its existence. That is to say, a system of material bodies with
spatial relation, or ``successions of situation'', is sufficient
and necessary in defining space as a relational entity.

Before starting, I would like to assume the naive substantial
distinction between the physical and the mental. The reason
being if the latter collapses into the former, then my premise
is true by definition. As a caveat, I shall not attempt here to
demonstrate why the former cannot collapse into the latter, but
I can claim such a view ultimately defeats the point of
answering this question.

First, if space, as abstracted from everything, is a relational
entity, then the propositional properties derived from a set of
sentences describing a system of bodies in space would be
relational properties. If ``I am next to the computer'' is such
a system of bodies, then ``being next to'' is the relational
property derived from the system. Now, if I remove the subject
and object particles from the sentence, and replace them with
$X$ and $Y$ respectively, then we have the sentence ``X is next
to $Y$''. If I attempt to substitute $X$ and $Y$ with something,
then I \textit{prima facie} do so with physical objects---those
comprises of matter. I claim that such an action is ostensible
proof that only objects of matter can substitute $X$ or $Y$ in
this sentence. While such a claim appear terse, consider the
opposing case using mental images. If I now sit in park and
think of an image of myself being next to my computer, then
while I can utter the sentence ``I am next to the computer'',
such a sentence will make no sense unless I clarify that ``the
concept of myself is next to the concept of a computer''. But
even then, such a sentence is meaningless. If two concepts can
be said to be next to one another, then they can also be said to
be on top, under, in front, etc., without any means to
differentiate between them---they are mere utterances.
Therefore, while it might be painfully trivial, if spatial
relations can be ascribed to both physical or mental bodies,
only the first will have any worthwhile propositional content.
This proves my first premise.

\section{Premise 2}%
\label{sec:premise_2}

I shall now argue that space is not a substantival entity. By
``substantival'', I mean that which is a substance, or in other
words, that which exists on its own. Being substantival
necessarily entails that matter is independent from space, by
definition. If space is not substantival, then it is relational
(I will not involve super-substanvialism in my discussion as
this involves showing matter is not a substance).

In order to do this, I will show that absolute space\ldots

\section{Conclusion}%
\label{sec:conclusion}

\printbibliography % chktex 1
\end{document}
